% Options for packages loaded elsewhere
\PassOptionsToPackage{unicode}{hyperref}
\PassOptionsToPackage{hyphens}{url}
%
\documentclass[
]{article}
\title{Time Series Analysis Module 1 Assignment}
\author{Filipp Krasovsky}
\date{}

\usepackage{amsmath,amssymb}
\usepackage{lmodern}
\usepackage{iftex}
\ifPDFTeX
  \usepackage[T1]{fontenc}
  \usepackage[utf8]{inputenc}
  \usepackage{textcomp} % provide euro and other symbols
\else % if luatex or xetex
  \usepackage{unicode-math}
  \defaultfontfeatures{Scale=MatchLowercase}
  \defaultfontfeatures[\rmfamily]{Ligatures=TeX,Scale=1}
\fi
% Use upquote if available, for straight quotes in verbatim environments
\IfFileExists{upquote.sty}{\usepackage{upquote}}{}
\IfFileExists{microtype.sty}{% use microtype if available
  \usepackage[]{microtype}
  \UseMicrotypeSet[protrusion]{basicmath} % disable protrusion for tt fonts
}{}
\makeatletter
\@ifundefined{KOMAClassName}{% if non-KOMA class
  \IfFileExists{parskip.sty}{%
    \usepackage{parskip}
  }{% else
    \setlength{\parindent}{0pt}
    \setlength{\parskip}{6pt plus 2pt minus 1pt}}
}{% if KOMA class
  \KOMAoptions{parskip=half}}
\makeatother
\usepackage{xcolor}
\IfFileExists{xurl.sty}{\usepackage{xurl}}{} % add URL line breaks if available
\IfFileExists{bookmark.sty}{\usepackage{bookmark}}{\usepackage{hyperref}}
\hypersetup{
  pdftitle={Time Series Analysis Module 1 Assignment},
  pdfauthor={Filipp Krasovsky},
  hidelinks,
  pdfcreator={LaTeX via pandoc}}
\urlstyle{same} % disable monospaced font for URLs
\usepackage[margin=1in]{geometry}
\usepackage{color}
\usepackage{fancyvrb}
\newcommand{\VerbBar}{|}
\newcommand{\VERB}{\Verb[commandchars=\\\{\}]}
\DefineVerbatimEnvironment{Highlighting}{Verbatim}{commandchars=\\\{\}}
% Add ',fontsize=\small' for more characters per line
\usepackage{framed}
\definecolor{shadecolor}{RGB}{248,248,248}
\newenvironment{Shaded}{\begin{snugshade}}{\end{snugshade}}
\newcommand{\AlertTok}[1]{\textcolor[rgb]{0.94,0.16,0.16}{#1}}
\newcommand{\AnnotationTok}[1]{\textcolor[rgb]{0.56,0.35,0.01}{\textbf{\textit{#1}}}}
\newcommand{\AttributeTok}[1]{\textcolor[rgb]{0.77,0.63,0.00}{#1}}
\newcommand{\BaseNTok}[1]{\textcolor[rgb]{0.00,0.00,0.81}{#1}}
\newcommand{\BuiltInTok}[1]{#1}
\newcommand{\CharTok}[1]{\textcolor[rgb]{0.31,0.60,0.02}{#1}}
\newcommand{\CommentTok}[1]{\textcolor[rgb]{0.56,0.35,0.01}{\textit{#1}}}
\newcommand{\CommentVarTok}[1]{\textcolor[rgb]{0.56,0.35,0.01}{\textbf{\textit{#1}}}}
\newcommand{\ConstantTok}[1]{\textcolor[rgb]{0.00,0.00,0.00}{#1}}
\newcommand{\ControlFlowTok}[1]{\textcolor[rgb]{0.13,0.29,0.53}{\textbf{#1}}}
\newcommand{\DataTypeTok}[1]{\textcolor[rgb]{0.13,0.29,0.53}{#1}}
\newcommand{\DecValTok}[1]{\textcolor[rgb]{0.00,0.00,0.81}{#1}}
\newcommand{\DocumentationTok}[1]{\textcolor[rgb]{0.56,0.35,0.01}{\textbf{\textit{#1}}}}
\newcommand{\ErrorTok}[1]{\textcolor[rgb]{0.64,0.00,0.00}{\textbf{#1}}}
\newcommand{\ExtensionTok}[1]{#1}
\newcommand{\FloatTok}[1]{\textcolor[rgb]{0.00,0.00,0.81}{#1}}
\newcommand{\FunctionTok}[1]{\textcolor[rgb]{0.00,0.00,0.00}{#1}}
\newcommand{\ImportTok}[1]{#1}
\newcommand{\InformationTok}[1]{\textcolor[rgb]{0.56,0.35,0.01}{\textbf{\textit{#1}}}}
\newcommand{\KeywordTok}[1]{\textcolor[rgb]{0.13,0.29,0.53}{\textbf{#1}}}
\newcommand{\NormalTok}[1]{#1}
\newcommand{\OperatorTok}[1]{\textcolor[rgb]{0.81,0.36,0.00}{\textbf{#1}}}
\newcommand{\OtherTok}[1]{\textcolor[rgb]{0.56,0.35,0.01}{#1}}
\newcommand{\PreprocessorTok}[1]{\textcolor[rgb]{0.56,0.35,0.01}{\textit{#1}}}
\newcommand{\RegionMarkerTok}[1]{#1}
\newcommand{\SpecialCharTok}[1]{\textcolor[rgb]{0.00,0.00,0.00}{#1}}
\newcommand{\SpecialStringTok}[1]{\textcolor[rgb]{0.31,0.60,0.02}{#1}}
\newcommand{\StringTok}[1]{\textcolor[rgb]{0.31,0.60,0.02}{#1}}
\newcommand{\VariableTok}[1]{\textcolor[rgb]{0.00,0.00,0.00}{#1}}
\newcommand{\VerbatimStringTok}[1]{\textcolor[rgb]{0.31,0.60,0.02}{#1}}
\newcommand{\WarningTok}[1]{\textcolor[rgb]{0.56,0.35,0.01}{\textbf{\textit{#1}}}}
\usepackage{graphicx}
\makeatletter
\def\maxwidth{\ifdim\Gin@nat@width>\linewidth\linewidth\else\Gin@nat@width\fi}
\def\maxheight{\ifdim\Gin@nat@height>\textheight\textheight\else\Gin@nat@height\fi}
\makeatother
% Scale images if necessary, so that they will not overflow the page
% margins by default, and it is still possible to overwrite the defaults
% using explicit options in \includegraphics[width, height, ...]{}
\setkeys{Gin}{width=\maxwidth,height=\maxheight,keepaspectratio}
% Set default figure placement to htbp
\makeatletter
\def\fps@figure{htbp}
\makeatother
\setlength{\emergencystretch}{3em} % prevent overfull lines
\providecommand{\tightlist}{%
  \setlength{\itemsep}{0pt}\setlength{\parskip}{0pt}}
\setcounter{secnumdepth}{-\maxdimen} % remove section numbering
\ifLuaTeX
  \usepackage{selnolig}  % disable illegal ligatures
\fi

\begin{document}
\maketitle

Textbook Exercises (Pages 14, 35)

\begin{verbatim}
## Loading required package: astsa
\end{verbatim}

1.1. {[}8 points{]} a) Generate n=100 observations from the
autoregression xt = -.9xt-2 + wt

\begin{Shaded}
\begin{Highlighting}[]
\CommentTok{\#create an AR{-}2 series with coefficients B1 = 0 and B2={-}0.9,}
\CommentTok{\#creating xt = 0(xt{-}1) {-} 0.9(xt{-}2) + wt}
\FunctionTok{set.seed}\NormalTok{(}\DecValTok{1002}\NormalTok{)}
\NormalTok{ar2 }\OtherTok{\textless{}{-}} \FunctionTok{arima.sim}\NormalTok{(}\FunctionTok{list}\NormalTok{(}\AttributeTok{order =} \FunctionTok{c}\NormalTok{(}\DecValTok{2}\NormalTok{,}\DecValTok{0}\NormalTok{,}\DecValTok{0}\NormalTok{), }\AttributeTok{ar =} \FunctionTok{c}\NormalTok{(}\DecValTok{0}\NormalTok{,}\SpecialCharTok{{-}}\FloatTok{0.9}\NormalTok{)), }\AttributeTok{n =} \DecValTok{100}\NormalTok{)}
\FunctionTok{ts.plot}\NormalTok{(ar2,}\AttributeTok{main=}\StringTok{"X\^{}t = {-}0.9X\^{}(t{-}2) + Wt"}\NormalTok{)}
\end{Highlighting}
\end{Shaded}

\includegraphics{Module1-Assignment_files/figure-latex/unnamed-chunk-2-1.pdf}
Apply a moving average filter (xt + xt-1 + xt-2 + xt-3) / 4 to the
series.

\begin{Shaded}
\begin{Highlighting}[]
\CommentTok{\#use the filter command to create a moving average}
\NormalTok{ma3 }\OtherTok{=} \FunctionTok{filter}\NormalTok{(ar2,}\AttributeTok{sides=}\DecValTok{1}\NormalTok{,}\AttributeTok{filter=}\FunctionTok{rep}\NormalTok{(}\DecValTok{1}\SpecialCharTok{/}\DecValTok{4}\NormalTok{,}\DecValTok{4}\NormalTok{))}
\CommentTok{\#plot and superimpose}
\FunctionTok{tsplot}\NormalTok{(ar2,}\AttributeTok{col=}\StringTok{"black"}\NormalTok{,}\AttributeTok{main=}\StringTok{"AR{-}2 vs. MA{-}3 transformation"}\NormalTok{)}
\FunctionTok{lines}\NormalTok{(ma3,}\AttributeTok{col=}\StringTok{"blue"}\NormalTok{,}\AttributeTok{lty=}\DecValTok{2}\NormalTok{)}
\FunctionTok{legend}\NormalTok{(}\AttributeTok{x=}\StringTok{"topleft"}\NormalTok{,}\AttributeTok{legend=}\FunctionTok{c}\NormalTok{(}\StringTok{"MA{-}3"}\NormalTok{,}\StringTok{"AR{-}2"}\NormalTok{),}\AttributeTok{col=}\FunctionTok{c}\NormalTok{(}\StringTok{"blue"}\NormalTok{,}\StringTok{"black"}\NormalTok{),}\AttributeTok{cex=}\FloatTok{0.8}\NormalTok{,}\AttributeTok{lty=}\DecValTok{2}\SpecialCharTok{:}\DecValTok{1}\NormalTok{)}
\end{Highlighting}
\end{Shaded}

\includegraphics{Module1-Assignment_files/figure-latex/unnamed-chunk-3-1.pdf}
Repeat with X\^{}t = 2cos(2pi * (t/4)) + wt.

\begin{Shaded}
\begin{Highlighting}[]
\CommentTok{\#First, create our signal + noise model as well as an index for time.}
\NormalTok{t }\OtherTok{=} \DecValTok{1}\SpecialCharTok{:}\DecValTok{100}
\NormalTok{sig }\OtherTok{=} \DecValTok{2} \SpecialCharTok{*} \FunctionTok{cos}\NormalTok{(}\DecValTok{2}\SpecialCharTok{*}\NormalTok{pi}\SpecialCharTok{*}\NormalTok{(t}\SpecialCharTok{/}\DecValTok{4}\NormalTok{))}
\NormalTok{noise }\OtherTok{=} \FunctionTok{rnorm}\NormalTok{(}\DecValTok{100}\NormalTok{)}
\NormalTok{signoise }\OtherTok{=}\NormalTok{ sig }\SpecialCharTok{+}\NormalTok{ noise }

\CommentTok{\#use the filter command to create a moving average}
\NormalTok{signoise\_ma3 }\OtherTok{=} \FunctionTok{filter}\NormalTok{(signoise,}\AttributeTok{sides=}\DecValTok{1}\NormalTok{,}\AttributeTok{filter=}\FunctionTok{rep}\NormalTok{(}\DecValTok{1}\SpecialCharTok{/}\DecValTok{4}\NormalTok{,}\DecValTok{4}\NormalTok{))}
\CommentTok{\#plot and superimpose}
\FunctionTok{tsplot}\NormalTok{(signoise,}\AttributeTok{col=}\StringTok{"black"}\NormalTok{,}\AttributeTok{main=}\StringTok{"Signal + Noise vs. MA{-}3 transformation"}\NormalTok{)}
\FunctionTok{lines}\NormalTok{(signoise\_ma3,}\AttributeTok{col=}\StringTok{"blue"}\NormalTok{,}\AttributeTok{lty=}\DecValTok{2}\NormalTok{)}
\FunctionTok{lines}\NormalTok{(sig,}\AttributeTok{col=}\StringTok{"gray"}\NormalTok{,}\AttributeTok{lty=}\DecValTok{3}\NormalTok{)}
\FunctionTok{legend}\NormalTok{(}\AttributeTok{x=}\StringTok{"topleft"}\NormalTok{,}\AttributeTok{legend=}\FunctionTok{c}\NormalTok{(}\StringTok{"MA{-}3"}\NormalTok{,}\StringTok{"Signal + Noise"}\NormalTok{,}\StringTok{"Signal"}\NormalTok{),}\AttributeTok{col=}\FunctionTok{c}\NormalTok{(}\StringTok{"blue"}\NormalTok{,}\StringTok{"black"}\NormalTok{,}\StringTok{"gray"}\NormalTok{),}\AttributeTok{cex=}\FloatTok{0.8}\NormalTok{,}\AttributeTok{lty=}\FunctionTok{c}\NormalTok{(}\DecValTok{2}\NormalTok{,}\DecValTok{1}\NormalTok{,}\DecValTok{3}\NormalTok{))}
\end{Highlighting}
\end{Shaded}

\includegraphics{Module1-Assignment_files/figure-latex/unnamed-chunk-4-1.pdf}
Repeat with the log of the johnson and johnson data in example 1.2 from
the textbook.

\begin{Shaded}
\begin{Highlighting}[]
\CommentTok{\#long{-}form syntax to make origin clear.}
\NormalTok{logjj }\OtherTok{=} \FunctionTok{log}\NormalTok{(astsa}\SpecialCharTok{::}\NormalTok{jj)}
\CommentTok{\#use the filter command to create a moving average}
\NormalTok{jj\_ma3 }\OtherTok{=} \FunctionTok{filter}\NormalTok{(logjj,}\AttributeTok{sides=}\DecValTok{1}\NormalTok{,}\AttributeTok{filter=}\FunctionTok{rep}\NormalTok{(}\DecValTok{1}\SpecialCharTok{/}\DecValTok{4}\NormalTok{,}\DecValTok{4}\NormalTok{))}
\CommentTok{\#plot and superimpose}
\FunctionTok{tsplot}\NormalTok{(logjj,}\AttributeTok{col=}\StringTok{"black"}\NormalTok{,}\AttributeTok{main=}\StringTok{"Log of J\&J Earnings vs. MA{-}3"}\NormalTok{)}
\FunctionTok{lines}\NormalTok{(jj\_ma3,}\AttributeTok{col=}\StringTok{"blue"}\NormalTok{,}\AttributeTok{lty=}\DecValTok{2}\NormalTok{)}
\FunctionTok{legend}\NormalTok{(}\AttributeTok{x=}\StringTok{"topleft"}\NormalTok{,}\AttributeTok{legend=}\FunctionTok{c}\NormalTok{(}\StringTok{"MA{-}3"}\NormalTok{,}\StringTok{"Log Earnings"}\NormalTok{),}\AttributeTok{col=}\FunctionTok{c}\NormalTok{(}\StringTok{"blue"}\NormalTok{,}\StringTok{"black"}\NormalTok{),}\AttributeTok{cex=}\FloatTok{0.8}\NormalTok{,}\AttributeTok{lty=}\FunctionTok{c}\NormalTok{(}\DecValTok{2}\NormalTok{,}\DecValTok{1}\NormalTok{))}
\end{Highlighting}
\end{Shaded}

\includegraphics{Module1-Assignment_files/figure-latex/unnamed-chunk-5-1.pdf}
What is seasonal adjustment? Seasonal adjusment is defined as a method
of data smoothing for economic performance over a given period, the
primary goal of which is to remove cyclical trends and provide a
nonseasonal view.

Conclusions: I learned how to generate simulated time series, and as a
corollary, how to identify when time series data is behaving in a way
that can be modeled as either an AR or MA process. Furthermore, I
learned that moving averages are able to smooth noisy data.

Exercise 1.3 - Working with Random Walk and Moving Average

Generate and plot nine series that are random walks of length n=300
without drift (theta=0) and variance = 1. Then, generate nine series of
length n = 500 that are moving averages of the form discussed in example
1.8

\begin{Shaded}
\begin{Highlighting}[]
\FunctionTok{par}\NormalTok{(}\AttributeTok{mfrow=}\FunctionTok{c}\NormalTok{(}\DecValTok{2}\NormalTok{,}\DecValTok{1}\NormalTok{))}
\ControlFlowTok{for}\NormalTok{ (i }\ControlFlowTok{in} \FunctionTok{c}\NormalTok{(}\DecValTok{1}\SpecialCharTok{:}\DecValTok{9}\NormalTok{))\{}
\NormalTok{  this.walk }\OtherTok{=} \FunctionTok{rnorm}\NormalTok{(}\DecValTok{300}\NormalTok{)}
\NormalTok{  this.walk }\OtherTok{=} \FunctionTok{cumsum}\NormalTok{(this.walk)}
  
  \ControlFlowTok{if}\NormalTok{(i}\SpecialCharTok{==}\DecValTok{1}\NormalTok{)\{}
    \FunctionTok{tsplot}\NormalTok{(this.walk,}\AttributeTok{main=}\StringTok{"Random Walks 1{-}9"}\NormalTok{,}\AttributeTok{col=}\NormalTok{i)}
\NormalTok{  \}}\ControlFlowTok{else}\NormalTok{\{}
    \FunctionTok{lines}\NormalTok{(this.walk,}\AttributeTok{col=}\NormalTok{i)}
\NormalTok{  \}}
\NormalTok{\}}

\ControlFlowTok{for}\NormalTok{(i }\ControlFlowTok{in} \FunctionTok{c}\NormalTok{(}\DecValTok{1}\SpecialCharTok{:}\DecValTok{9}\NormalTok{))\{}
\NormalTok{  w }\OtherTok{=} \FunctionTok{rnorm}\NormalTok{(}\DecValTok{500}\NormalTok{)}
\NormalTok{  this.ma }\OtherTok{=} \FunctionTok{filter}\NormalTok{(w,}\AttributeTok{sides=}\DecValTok{2}\NormalTok{,}\AttributeTok{filter=}\FunctionTok{rep}\NormalTok{(}\DecValTok{1}\SpecialCharTok{/}\DecValTok{3}\NormalTok{,}\DecValTok{3}\NormalTok{))}
  \ControlFlowTok{if}\NormalTok{(i}\SpecialCharTok{==}\DecValTok{1}\NormalTok{)\{}
    \FunctionTok{tsplot}\NormalTok{(this.ma,}\AttributeTok{main=}\StringTok{"Moving Avg. 1{-}9"}\NormalTok{,}\AttributeTok{col=}\NormalTok{i)}
\NormalTok{  \}}\ControlFlowTok{else}\NormalTok{\{}
    \FunctionTok{lines}\NormalTok{(this.ma,}\AttributeTok{col=}\NormalTok{i)}
\NormalTok{  \}}
\NormalTok{\}}
\end{Highlighting}
\end{Shaded}

\includegraphics{Module1-Assignment_files/figure-latex/unnamed-chunk-6-1.pdf}

The primary differences between the two types of time series are that
Random Walks are less stationary than moving averages. More importantly,
the variance of a random walk increases without bound as time increases,
whereas the variance of a moving average does not.

1.4. {[}7 points{]} The data in gdp are the seasonally adjusted
quarterly U.S. GDP from 1947-I to 2018-III. The growth rate is shown in
Figure 1.4. a) Plot the data and compare it to one of the models
discussed in Section 1.3.

\begin{Shaded}
\begin{Highlighting}[]
\CommentTok{\#we reference the gdp dataset from astsa::gdp }
\CommentTok{\#plot the data }
\FunctionTok{tsplot}\NormalTok{(astsa}\SpecialCharTok{::}\NormalTok{gdp, }\AttributeTok{main=}\StringTok{"Seasonally Adjusted GDP data, 1947Q1{-}2018Q3"}\NormalTok{)}
\end{Highlighting}
\end{Shaded}

\includegraphics{Module1-Assignment_files/figure-latex/unnamed-chunk-7-1.pdf}
b) Reproduce Figure 1.4 using your colors and plot characters (pch) of
your own choice. Then, comment on the difference between the two methods
of calculating growth rate.

\begin{Shaded}
\begin{Highlighting}[]
\NormalTok{logdiff }\OtherTok{=} \FunctionTok{diff}\NormalTok{(}\FunctionTok{log}\NormalTok{(gdp))}
\NormalTok{returns }\OtherTok{=} \FunctionTok{diff}\NormalTok{(gdp)}\SpecialCharTok{/}\FunctionTok{lag}\NormalTok{(gdp,}\SpecialCharTok{{-}}\DecValTok{1}\NormalTok{)}

\FunctionTok{tsplot}\NormalTok{(logdiff,}\AttributeTok{type=}\StringTok{"o"}\NormalTok{,}\AttributeTok{col=}\StringTok{"gray"}\NormalTok{,}\AttributeTok{ylab=}\StringTok{"GDP Growth"}\NormalTok{)}
\FunctionTok{points}\NormalTok{(returns,}\AttributeTok{pch=}\DecValTok{1}\NormalTok{,}\AttributeTok{col=}\DecValTok{2}\NormalTok{)}
\end{Highlighting}
\end{Shaded}

\includegraphics{Module1-Assignment_files/figure-latex/unnamed-chunk-8-1.pdf}
The output itself is negligibly different - however, the process differs
in that the first method uses a log transformation and takes successive
differences while the other calculates the percentage difference by
computing r = (xt - x(t-1))/x(t-1)

\begin{enumerate}
\def\labelenumi{\alph{enumi})}
\setcounter{enumi}{2}
\tightlist
\item
  Which of the models discussed in Section 1.3 best describe the
  behavior of the growth in U.S. GDP?
\end{enumerate}

\begin{Shaded}
\begin{Highlighting}[]
\CommentTok{\#plot the ACF and PACF to identify if this is AR or MA.}
\FunctionTok{acf}\NormalTok{(logdiff)}
\end{Highlighting}
\end{Shaded}

\includegraphics{Module1-Assignment_files/figure-latex/unnamed-chunk-9-1.pdf}

\begin{Shaded}
\begin{Highlighting}[]
\FunctionTok{pacf}\NormalTok{(logdiff)}
\end{Highlighting}
\end{Shaded}

\includegraphics{Module1-Assignment_files/figure-latex/unnamed-chunk-9-2.pdf}
Our correlograms show a gradually decreasing ACF and a PACF that cuts
after one spike. As a result, we can make the argument this is an AR(1)
model.

2.11. {[}7 points{]}

\begin{enumerate}
\def\labelenumi{\alph{enumi})}
\tightlist
\item
  Simulate a series of n=500 moving average observations as in Example
  1.8 and compute the sample ACF to lag 20. Compare the sample ACF you
  obtain to the actual ACF {[}Hint: Recall Example 2.17{]}
\end{enumerate}

\begin{Shaded}
\begin{Highlighting}[]
\CommentTok{\#reproduce an MA of n=500}
\NormalTok{w }\OtherTok{=} \FunctionTok{rnorm}\NormalTok{(}\DecValTok{500}\NormalTok{)}
\NormalTok{v }\OtherTok{=} \FunctionTok{filter}\NormalTok{(w,}\AttributeTok{sides=}\DecValTok{2}\NormalTok{,}\AttributeTok{filter=}\FunctionTok{rep}\NormalTok{(}\DecValTok{1}\SpecialCharTok{/}\DecValTok{3}\NormalTok{,}\DecValTok{3}\NormalTok{))}

\CommentTok{\#compute the sample ACF to lag twenty.}
\CommentTok{\#we can use the sample ACF formula:}
\CommentTok{\#sigma(1,n{-}h)\{(x\^{}t+h {-} mean(x)) * (x\^{}t {-} mean(x))\} / sigma(1,n)\{(x\^{}t{-}mean(x))\^{}2\}}
\CommentTok{\#...or we can just use the library.}

\FunctionTok{acf}\NormalTok{(w,}\AttributeTok{lag.max =} \DecValTok{20}\NormalTok{)}
\end{Highlighting}
\end{Shaded}

\includegraphics{Module1-Assignment_files/figure-latex/unnamed-chunk-10-1.pdf}
The actual ACF of an MA series would have gradually decreasing values as
the absolute lag moves from 0 to 2, and then becomes zero after any lags
\textgreater{} 2. In this instance, we only have a strong spike at zero.

\begin{enumerate}
\def\labelenumi{\alph{enumi})}
\setcounter{enumi}{1}
\tightlist
\item
  Repeat part(a) using only n=50. How does changing n affect the
  results?
\end{enumerate}

\begin{Shaded}
\begin{Highlighting}[]
\CommentTok{\#reproduce an MA of n=50}
\NormalTok{w }\OtherTok{=} \FunctionTok{rnorm}\NormalTok{(}\DecValTok{50}\NormalTok{)}
\NormalTok{v }\OtherTok{=} \FunctionTok{filter}\NormalTok{(w,}\AttributeTok{sides=}\DecValTok{2}\NormalTok{,}\AttributeTok{filter=}\FunctionTok{rep}\NormalTok{(}\DecValTok{1}\SpecialCharTok{/}\DecValTok{3}\NormalTok{,}\DecValTok{3}\NormalTok{))}

\CommentTok{\#compute the sample ACF to lag twenty.}
\CommentTok{\#we can use the sample ACF formula:}
\CommentTok{\#sigma(1,n{-}h)\{(x\^{}t+h {-} mean(x)) * (x\^{}t {-} mean(x))\} / sigma(1,n)\{(x\^{}t{-}mean(x))\^{}2\}}
\CommentTok{\#...or we can just use the library.}

\FunctionTok{acf}\NormalTok{(w,}\AttributeTok{lag.max =} \DecValTok{20}\NormalTok{)}
\end{Highlighting}
\end{Shaded}

\includegraphics{Module1-Assignment_files/figure-latex/unnamed-chunk-11-1.pdf}
The key differences are rooted in (1) a higher standard for
significance, (2) changing magnitudes for spikes, and (3) a change in
directionality of the lag spikes. Making n smaller creates more
variability in the ACF.

2.12. {[}7 points{]}

Simulate 500 observations from the AR model specified in Example 1.9 and
then plot the sample ACF to lag 50. What does the sample ACF tell you
about the approximate cyclic behavior of the data? {[}Hint: Recall
Example 2.32{]}

\begin{Shaded}
\begin{Highlighting}[]
\FunctionTok{set.seed}\NormalTok{(}\DecValTok{90210}\NormalTok{)  }
\NormalTok{w }\OtherTok{=} \FunctionTok{rnorm}\NormalTok{(}\DecValTok{500} \SpecialCharTok{+} \DecValTok{50}\NormalTok{) }
\CommentTok{\#50 extra to avoid startup problems  }
\NormalTok{x }\OtherTok{=} \FunctionTok{filter}\NormalTok{(w, }\AttributeTok{filter=}\FunctionTok{c}\NormalTok{(}\FloatTok{1.5}\NormalTok{,}\SpecialCharTok{{-}}\NormalTok{.}\DecValTok{75}\NormalTok{), }\AttributeTok{method=}\StringTok{"recursive"}\NormalTok{)[}\SpecialCharTok{{-}}\NormalTok{(}\DecValTok{1}\SpecialCharTok{:}\DecValTok{50}\NormalTok{)]  }
\FunctionTok{tsplot}\NormalTok{(x, }\AttributeTok{main=}\StringTok{"autoregression"}\NormalTok{, }\AttributeTok{col=}\DecValTok{4}\NormalTok{)}
\end{Highlighting}
\end{Shaded}

\includegraphics{Module1-Assignment_files/figure-latex/unnamed-chunk-12-1.pdf}

\begin{Shaded}
\begin{Highlighting}[]
\FunctionTok{acf}\NormalTok{(x,}\AttributeTok{lag.max =} \DecValTok{50}\NormalTok{)}
\end{Highlighting}
\end{Shaded}

\includegraphics{Module1-Assignment_files/figure-latex/unnamed-chunk-12-2.pdf}
Lags eventually decrease in siginificance after k = 30; however, we can
confirm that observations 12 moments in time apart are positively
correlated. Moments that are separated by around six lags tend to be
negatively correlated. Positive values of the AR process tend to be
associated with negative values 6 lags apart.

2.14. {[}7 points{]} Simulate a series of n=500 observations from the
signal-plus-noise model presented in Example 1.11 with standard
deviations of 0,1,and 5.Compute the sample ACF to lag 100 of the three
series you generated and comment.

\begin{Shaded}
\begin{Highlighting}[]
\NormalTok{t}\OtherTok{=}\DecValTok{500}
\ControlFlowTok{for}\NormalTok{ ( i }\ControlFlowTok{in} \FunctionTok{c}\NormalTok{(}\DecValTok{0}\NormalTok{,}\DecValTok{1}\NormalTok{,}\DecValTok{5}\NormalTok{))\{}
\NormalTok{  cs }\OtherTok{=} \DecValTok{2}\SpecialCharTok{*}\FunctionTok{cos}\NormalTok{(}\DecValTok{2}\SpecialCharTok{*}\NormalTok{pi}\SpecialCharTok{*}\NormalTok{(t}\SpecialCharTok{+}\DecValTok{15}\NormalTok{)}\SpecialCharTok{/}\DecValTok{50}\NormalTok{)}
\NormalTok{  w }\OtherTok{=} \FunctionTok{rnorm}\NormalTok{(}\AttributeTok{n=}\DecValTok{500}\NormalTok{,}\AttributeTok{mean=}\DecValTok{0}\NormalTok{,}\AttributeTok{sd=}\NormalTok{i)}
\NormalTok{  x }\OtherTok{=}\NormalTok{ cs }\SpecialCharTok{+}\NormalTok{ w }
\NormalTok{  x[}\DecValTok{500}\NormalTok{]}\OtherTok{=}\FloatTok{0.00001}
  \CommentTok{\#tsplot(sn,main=paste("Signal+Noise with Sigma:",i))}
  \FunctionTok{acf}\NormalTok{(x,}\AttributeTok{main=}\FunctionTok{paste}\NormalTok{(}\StringTok{"Signal + Noise ACF w/ Sigma"}\NormalTok{,i),}\AttributeTok{lag.max =} \DecValTok{100}\NormalTok{)}
\NormalTok{\}}
\end{Highlighting}
\end{Shaded}

\includegraphics{Module1-Assignment_files/figure-latex/unnamed-chunk-13-1.pdf}
\includegraphics{Module1-Assignment_files/figure-latex/unnamed-chunk-13-2.pdf}
\includegraphics{Module1-Assignment_files/figure-latex/unnamed-chunk-13-3.pdf}
Remarks: The first time series has no autocorrelative properties because
the value is a constant with no variance. As the standard deviation
becomes larger, the number of significant lag spikes increases as well -
such as in the case of the final series, with potentially significance
spikes at lag \textasciitilde60 and \textasciitilde90.

\end{document}
